\documentclass[a4paper,10pt,twoside]{article}

\usepackage[top=1in, bottom=1in, left=1in, right=1in]{geometry}
\usepackage[utf8]{inputenc}
\usepackage[spanish,es-ucroman,es-noquoting]{babel}
\usepackage{setspace}
\usepackage{fancyhdr}
\usepackage{lastpage}
\usepackage{amsmath}
\usepackage{amsfonts}
\usepackage{verbatim}
\usepackage{graphicx}
\usepackage{float}
\usepackage{algpseudocode}
\usepackage{enumitem} % Provee macro \setlist
\usepackage[toc, page]{appendix}


%%%%%%%%%% Configuración de Fancyhdr - Inicio %%%%%%%%%%
\pagestyle{fancy}
\thispagestyle{fancy}
\lhead{Trabajo Práctico 3 · Sistemas Operativos}
\rhead{Delgado · Lovisolo · Petaccio}
\renewcommand{\footrulewidth}{0.4pt}
\cfoot{\thepage /\pageref{LastPage}}

\fancypagestyle{caratula} {
   \fancyhf{}
   \cfoot{\thepage /\pageref{LastPage}}
   \renewcommand{\headrulewidth}{0pt}
   \renewcommand{\footrulewidth}{0pt}
}
%%%%%%%%%% Configuración de Fancyhdr - Fin %%%%%%%%%%


%%%%%%%%%% Configuración de Algorithmic - Inicio %%%%%%%%%%
% Entorno propio para customizar la presentación del pseudocódigo
\newenvironment{pseudo}[1][]{%
    \vspace{0.5em}%
    \begin{algorithmic}%
}
{%
    \end{algorithmic}%
    \vspace{0.5em}%
}

% Llamada a función para usar así: \Fn{Foo}{bar, baz}.
% Produce \textsc{Foo}$(bar, baz)$.
\newcommand{\Fn}[2]{\textsc{#1}$(#2)$}

% Cláusula return para usar así: \PReturn foo
\newcommand{\PReturn}[1]{\textbf{return} $#1$}

% Cláusula break
\newcommand{\Break}{\textbf{break}}

% Operadores lógicos
\newcommand{\PAnd}{\textbf{and} }
\newcommand{\POr}{\textbf{or} }
\newcommand{\PNot}{\textbf{not} }

% Booleanos
\newcommand{\PTrue}{\textnormal{TRUE} }
\newcommand{\PFalse}{\textnormal{FALSE} }

% Null
\newcommand{\Null}{\textnormal{NULL} }

% Conectivo 'in' para usar así: \ForAll{$foo$ \In $bar$}
\newcommand{\In}{\textbf{in} }

% Conectivo 'to' para usar así: \For{$i = 1$ \In $n$}
\newcommand{\To}{\textbf{to} }

% Complejidades
\newcommand{\Ode}[1]{\hfill $O(#1)$}
%%%%%%%%%% Configuración de Algorithmic - Fin %%%%%%%%%%


%%%%%%%%%% Configuración de Appendix - Inicio %%%%%%%%%%
% Asigna la traducción de la palabra 'Appendices'.
\renewcommand{\appendixtocname}{Apéndices}
\renewcommand{\appendixpagename}{Apéndices}
%%%%%%%%%% Configuración de Appendix - Fin %%%%%%%%%%


%%%%%%%%%% Miscelánea - Inicio %%%%%%%%%%
% Evita que el documento se estire verticalmente para ocupar el espacio vacío
% en cada página.
\raggedbottom

% Deshabilita sangría en la primer línea de un párrafo.
\setlength{\parindent}{0em}

% Separación entre párrafos.
\setlength{\parskip}{0.5em}

% Separación entre elementos de listas.
\setlist{itemsep=0.5em}
%%%%%%%%%% Miscelánea - Fin %%%%%%%%%%


\begin{document}


%%%%%%%%%%%%%%%%%%%%%%%%%%%%%%%%%%%%%%%%%%%%%%%%%%%%%%%%%%%%%%%%%%%%%%%%%%%%%%%
%% Carátula                                                                  %%
%%%%%%%%%%%%%%%%%%%%%%%%%%%%%%%%%%%%%%%%%%%%%%%%%%%%%%%%%%%%%%%%%%%%%%%%%%%%%%%


\thispagestyle{caratula}

\begin{center}

\includegraphics[height=2cm]{DC.png} 
\hfill
\includegraphics[height=2cm]{UBA.jpg} 

\vspace{2cm}

Departamento de Computación,\\
Facultad de Ciencias Exactas y Naturales,\\
Universidad de Buenos Aires

\vspace{4cm}

\begin{Huge}
Trabajo Práctico 3
\end{Huge}

\vspace{0.5cm}

\begin{Large}
Sistemas Operativos
\end{Large}

\vspace{1cm}

Segundo Cuatrimestre de 2013

\vspace{4cm}

\begin{tabular}{|c|c|c|}
\hline
Apellido y Nombre & LU & E-mail\\
\hline
Alejandro Nahuel Delgado & 601/11 & nahueldelgado@gmail.com\\
Leandro Lovisolo         & 645/11 & leandro@leandro.me\\
Lautaro José Petaccio    & 443/11 & lausuper@gmail.com\\
\hline
\end{tabular}

\end{center}

\newpage


%%%%%%%%%%%%%%%%%%%%%%%%%%%%%%%%%%%%%%%%%%%%%%%%%%%%%%%%%%%%%%%%%%%%%%%%%%%%%%%
%% Índice                                                                    %%
%%%%%%%%%%%%%%%%%%%%%%%%%%%%%%%%%%%%%%%%%%%%%%%%%%%%%%%%%%%%%%%%%%%%%%%%%%%%%%%


\tableofcontents

\newpage


%%%%%%%%%%%%%%%%%%%%%%%%%%%%%%%%%%%%%%%%%%%%%%%%%%%%%%%%%%%%%%%%%%%%%%%%%%%%%%%
%% Análisis del cuerpo de datos                                              %%
%%%%%%%%%%%%%%%%%%%%%%%%%%%%%%%%%%%%%%%%%%%%%%%%%%%%%%%%%%%%%%%%%%%%%%%%%%%%%%%


\section{Análisis del cuerpo de datos}

\newcommand{\map}{\emph{map} }
\newcommand{\reduce}{\emph{reduce} }


\subsection{¿Es la comunidad en promedio más \emph{upvoter} o más \emph{downvoter}?}

En el paso \map se lleva cada submission a una estructura con los siguientes campos:

\begin{description}
	\item[upvotes:] Cantidad de votos positivos.
	\item[total\_upvotes:] Cantidad total de votos.
	\item[positivity:] Campo usado en el paso \reduce. Entero con valor 0.
\end{description}

En el paso \reduce se acumula la suma de todos los votos positivos en el campo \textbf{upvotes}, se acumula el total de los votos en el campo \textbf{total\_upvotes} y se calcula el porcentaje de votos positivos en el campo \textbf{positivity}, de acuerdo a la fórmula

$$\textbf{positivity} = \frac{\textbf{upvotes} \times 100}{\textbf{total\_upvotes}}.$$

La comunidad se determina más \emph{upvoter} si $\textbf{positivity} > 50$, o \emph{downvoter} en caso contrario.


\subsection{Promedio de comentarios por submission}

En el paso \map se lleva cada submission a una estructura con los siguientes campos:

\begin{description}
	\item[comments:] Cantidad de comentarios de la submission.
	\item[submissions:] Campo usado en el paso \reduce para acumular la cantidad de submissions en el cuerpo de datos. Entero con valor 1.
	\item[avgComments:] Campo usado en el paso \reduce para calcular el promedo de comentarios por submission. Entero con valor 0.
\end{description}

En el paso \reduce se acumula el total de comentarios en el campo \textbf{comments}, se acumula el total de submissions en el campo \textbf{submissions} y se calcula el promedio de comentarios por submission en el campo \textbf{avgComments}, de acuerdo a la fórmula

$$\textbf{avgComments} = \frac{\textbf{comments}}{\textbf{submissions}}.$$

El promedio de comentarios por submission en la comunidad es entonces el valor del campo \textbf{avgComments}.

%%%%%%%%%%%%%%%%%%%%%%%%%%%%%%%%%%%%%%%%%%%%%%%%%%%%%%%%%%%%%%%%%%%%%%%%%%%%%%%
%% Análisis de escalabilidad                                                 %%
%%%%%%%%%%%%%%%%%%%%%%%%%%%%%%%%%%%%%%%%%%%%%%%%%%%%%%%%%%%%%%%%%%%%%%%%%%%%%%%

\section{Análisis de escalabilidad}
\subsection{Definición de \emph{Cloud Computing}}

\subsection{Alternativas de \emph{IaaS} y \emph{PaaS}}

\subsection{De la nube al hardware propio}


\end{document}