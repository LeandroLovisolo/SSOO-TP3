\documentclass[a4paper,10pt,twoside]{article}

\usepackage[top=1in, bottom=1in, left=1in, right=1in]{geometry}
\usepackage[utf8]{inputenc}
\usepackage[spanish,es-ucroman,es-noquoting]{babel}
\usepackage{setspace}
\usepackage{xspace}
\usepackage{fancyhdr}
\usepackage{lastpage}
\usepackage{amsmath}
\usepackage{amsfonts}
\usepackage{verbatim}
\usepackage{graphicx}
\usepackage{float}
\usepackage{algpseudocode}
\usepackage{enumitem} % Provee macro \setlist
\usepackage[toc, page]{appendix}


%%%%%%%%%% Configuración de Fancyhdr - Inicio %%%%%%%%%%
\pagestyle{fancy}
\thispagestyle{fancy}
\lhead{Trabajo Práctico 3 · Sistemas Operativos}
\rhead{Delgado · Lovisolo · Petaccio}
\renewcommand{\footrulewidth}{0.4pt}
\cfoot{\thepage /\pageref{LastPage}}

\fancypagestyle{caratula} {
   \fancyhf{}
   \cfoot{\thepage /\pageref{LastPage}}
   \renewcommand{\headrulewidth}{0pt}
   \renewcommand{\footrulewidth}{0pt}
}
%%%%%%%%%% Configuración de Fancyhdr - Fin %%%%%%%%%%


%%%%%%%%%% Configuración de Algorithmic - Inicio %%%%%%%%%%
% Entorno propio para customizar la presentación del pseudocódigo
\newenvironment{pseudo}[1][]{%
    \vspace{0.5em}%
    \begin{algorithmic}%
}
{%
    \end{algorithmic}%
    \vspace{0.5em}%
}

% Llamada a función para usar así: \Fn{Foo}{bar, baz}.
% Produce \textsc{Foo}$(bar, baz)$.
\newcommand{\Fn}[2]{\textsc{#1}$(#2)$}

% Cláusula return para usar así: \PReturn foo
\newcommand{\PReturn}[1]{\textbf{return} $#1$}

% Cláusula break
\newcommand{\Break}{\textbf{break}}

% Operadores lógicos
\newcommand{\PAnd}{\textbf{and} }
\newcommand{\POr}{\textbf{or} }
\newcommand{\PNot}{\textbf{not} }

% Booleanos
\newcommand{\PTrue}{\textnormal{TRUE} }
\newcommand{\PFalse}{\textnormal{FALSE} }

% Null
\newcommand{\Null}{\textnormal{NULL} }

% Conectivo 'in' para usar así: \ForAll{$foo$ \In $bar$}
\newcommand{\In}{\textbf{in} }

% Conectivo 'to' para usar así: \For{$i = 1$ \In $n$}
\newcommand{\To}{\textbf{to} }

% Complejidades
\newcommand{\Ode}[1]{\hfill $O(#1)$}
%%%%%%%%%% Configuración de Algorithmic - Fin %%%%%%%%%%


%%%%%%%%%% Configuración de Appendix - Inicio %%%%%%%%%%
% Asigna la traducción de la palabra 'Appendices'.
\renewcommand{\appendixtocname}{Apéndices}
\renewcommand{\appendixpagename}{Apéndices}
%%%%%%%%%% Configuración de Appendix - Fin %%%%%%%%%%


%%%%%%%%%% Miscelánea - Inicio %%%%%%%%%%
% Evita que el documento se estire verticalmente para ocupar el espacio vacío
% en cada página.
\raggedbottom

% Deshabilita sangría en la primer línea de un párrafo.
\setlength{\parindent}{0em}

% Separación entre párrafos.
\setlength{\parskip}{0.5em}

% Separación entre elementos de listas.
\setlist{itemsep=0.5em}
%%%%%%%%%% Miscelánea - Fin %%%%%%%%%%


%%%%%%%%%% Macros para el análisis de las consultas - Inicio %%%%%%%%%%
\newcommand{\map}{\emph{map}\xspace}
\newcommand{\reduce}{\emph{reduce}\xspace}
%%%%%%%%%% Macros para el análisis de las consultas - Fin %%%%%%%%%%


\begin{document}


%%%%%%%%%%%%%%%%%%%%%%%%%%%%%%%%%%%%%%%%%%%%%%%%%%%%%%%%%%%%%%%%%%%%%%%%%%%%%%%
%% Carátula                                                                  %%
%%%%%%%%%%%%%%%%%%%%%%%%%%%%%%%%%%%%%%%%%%%%%%%%%%%%%%%%%%%%%%%%%%%%%%%%%%%%%%%


\thispagestyle{caratula}

\begin{center}

\includegraphics[height=2cm]{DC.png} 
\hfill
\includegraphics[height=2cm]{UBA.jpg} 

\vspace{2cm}

Departamento de Computación,\\
Facultad de Ciencias Exactas y Naturales,\\
Universidad de Buenos Aires

\vspace{4cm}

\begin{Huge}
Trabajo Práctico 3
\end{Huge}

\vspace{0.5cm}

\begin{Large}
Sistemas Operativos
\end{Large}

\vspace{1cm}

Segundo Cuatrimestre de 2013

\vspace{4cm}

\begin{tabular}{|c|c|c|}
\hline
Apellido y Nombre & LU & E-mail\\
\hline
Alejandro Nahuel Delgado & 601/11 & nahueldelgado@gmail.com\\
Leandro Lovisolo         & 645/11 & leandro@leandro.me\\
Lautaro José Petaccio    & 443/11 & lausuper@gmail.com\\
\hline
\end{tabular}

\end{center}

\newpage


%%%%%%%%%%%%%%%%%%%%%%%%%%%%%%%%%%%%%%%%%%%%%%%%%%%%%%%%%%%%%%%%%%%%%%%%%%%%%%%
%% Índice                                                                    %%
%%%%%%%%%%%%%%%%%%%%%%%%%%%%%%%%%%%%%%%%%%%%%%%%%%%%%%%%%%%%%%%%%%%%%%%%%%%%%%%


\tableofcontents

\newpage


%%%%%%%%%%%%%%%%%%%%%%%%%%%%%%%%%%%%%%%%%%%%%%%%%%%%%%%%%%%%%%%%%%%%%%%%%%%%%%%
%% Análisis del cuerpo de datos                                              %%
%%%%%%%%%%%%%%%%%%%%%%%%%%%%%%%%%%%%%%%%%%%%%%%%%%%%%%%%%%%%%%%%%%%%%%%%%%%%%%%


\section{Análisis del cuerpo de datos}


\subsection{¿Es la comunidad en promedio más \emph{upvoter} o más \emph{downvoter}?}

En el paso \map se emite una misma key para cada submission, asociada a una estructura con los siguientes campos:

\begin{description}
	\item[upvotes:] Cantidad de votos positivos.
	\item[total\_upvotes:] Cantidad total de votos.
	\item[positivity:] Campo usado en el paso \reduce. Entero con valor 0.
\end{description}

En el paso \reduce se acumula la suma de todos los votos positivos en el campo \textbf{upvotes}, se acumula el total de los votos en el campo \textbf{total\_upvotes} y se calcula el porcentaje de votos positivos en el campo \textbf{positivity}, de acuerdo a la fórmula

$$\textbf{positivity} = \frac{\textbf{upvotes} \times 100}{\textbf{total\_upvotes}}.$$

La comunidad se determina más \emph{upvoter} si $\textbf{positivity} > 50$, o \emph{downvoter} en caso contrario.


\subsection{Promedio de comentarios por submission}

En el paso \map se emite una misma key para cada submission, asociada a una estructura con los siguientes campos:

\begin{description}
	\item[comments:] Cantidad de comentarios de la submission.
	\item[submissions:] Campo usado en el paso \reduce para acumular la cantidad de submissions en el cuerpo de datos. Entero con valor 1.
	\item[avgComments:] Campo usado en el paso \reduce para calcular el promedo de comentarios por submission. Entero con valor 0.
\end{description}

En el paso \reduce se acumula el total de comentarios en el campo \textbf{comments}, se acumula el total de submissions en el campo \textbf{submissions} y se calcula el promedio de comentarios por submission en el campo \textbf{avgComments}, de acuerdo a la fórmula

$$\textbf{avgComments} = \frac{\textbf{comments}}{\textbf{submissions}}.$$

El promedio de comentarios por submission en la comunidad es entonces el valor del campo \textbf{avgComments}.


\subsection{El usuario con la suma (score) más alta de todos.}

Esta consulta devuelve el usuario tal que la suma del campo \textbf{score} de todas las submissions cuyo autor sea dicho usuario es máxima.

Se divide la consulta en dos pares de operaciones \map/\reduce.

En el primer \map se emite una key cuyo valor es el autor de la submission, asociada a una estructura con los siguientes campos:

\begin{description}
	\item[username:] El autor de la submission.
	\item[score:] El score de la submission.
\end{description}

Si el usuario de la submission es nulo, la función \map termina sin emitir ningún valor.

En el primer \reduce se acumula el score total de las submissions del usuario correspondiente a cada key bajo el campo \textbf{score}.

Luego del primer \map/\reduce se obtiene una colección con campos \textbf{username} y \textbf{score}, que para cada usuario nos indica la suma de los scores de todas las submissions cuyo autor es ese usuario.

El segundo \map agrupa bajo un mismo key todas las entradas de la colección obtenida en el \map/\reduce anterior, dejando los campos \textbf{username} y \textbf{score} intactos.

El segundo \reduce busca la entrada con máximo score y devuelve esa entrada sin modificarla.

El resultado del segundo \map/\reduce es una estructura con campos \textbf{username} y \textbf{score} que indica el usuario con la máxima suma de scores de submissions y el valor de dicha suma.


\subsection{Subreddit más popular}

Esta consulta devuelve el subreddit con el mayor número de comentarios; es decir, el subreddit tal que la suma de los comentarios de todas las submissions dentro de ese subreddit es máxima.

En el primer \map se emite una key cuyo valor es el subreddit de la submission, asociada a una estructura con los siguientes campos:

\begin{description}
	\item[subreddit:] El subreddit de la submission.
	\item[number\_of\_comments:] El número de comentarios de la submission.
\end{description}

En el primer \reduce se acumula el total de comentarios de las submissions en el subreddit correspondiente a cada key bajo el campo \textbf{number\_of\_comments}.

Luego del primer \map/\reduce se obtiene una colección con campos \textbf{subreddit} y \textbf{number\_of\_comments}, que para cada subreddit nos indica la suma del número de comentarios de todas las submissions dentro de ese subreddit.

El segundo \map agrupa bajo un mismo key todas las entradas de la colección obtenida en el \map/\reduce anterior, dejando los campos \textbf{subreddit} y \textbf{number\_of\_comments} intactos.

El segundo \reduce busca la entrada con máximo número de comentarios y devuelve esa entrada sin modificarla.

El resultado del segundo \map/\reduce es una estructura con campos \textbf{subreddit} y \textbf{number\_of\_comments} que indica el subreddit con la máxima cantidad de comentarios realizados en submissions dentro de ese subreddit y el valor de dicha suma.


%%%%%%%%%%%%%%%%%%%%%%%%%%%%%%%%%%%%%%%%%%%%%%%%%%%%%%%%%%%%%%%%%%%%%%%%%%%%%%%
%% Análisis de escalabilidad                                                 %%
%%%%%%%%%%%%%%%%%%%%%%%%%%%%%%%%%%%%%%%%%%%%%%%%%%%%%%%%%%%%%%%%%%%%%%%%%%%%%%%


\section{Análisis de escalabilidad}

Antes de comenzar con el análisis en sí de las soluciones disponibles en el mercado, veamos algunas definiciones necesarias para comprender y clasificar las opciones.


\subsection{Introducción}

Cloud Computing es un término muy discutido durante los últimos años, y la cantidad de servicios que lo proveen aumenta año tras año. Muchas empresas han incursionado en este modelo de negocio, y muchas más han sido las que han aprovechado su potencial. Aún así, no hay ningún tipo de estandarización de los servicios ni de la definición de los mismos conceptos. En principio vamos a tratar de establecer algunas guías y definiciones, para luego poder centrarnos en nuestras necesidades particulares alrededor de MongoDB.


\subsection{Definición de \emph{Cloud Computing}}

Una nube consiste en un conjunto de computadoras conectadas y virtualizadas que son dinámicamente proveídas y presentadas como uno o más recursos computacionales unificados. Los clientes se conectan a esta nube para poder hacer uso del servicio que se provee.

En general, los servicios de cloud computing comprenden lo siguiente:

\begin{itemize}
	\item Abstracción de la infraestructura y software subyacente para ser ofrecidos como un servicio.
	\item Mantenido sobre una infraestructura escalable y flexible.
	\item Ofrece el servicio a demanda y garantías de calidad de servicio.
	\item Se abona por el uso de recursos computacionales sin necesidad de que los usuarios se comprometan por adelantado.
	\item Compartido y multiusuario.
	\item Accesible a través de internet desde cualquier dispositivo.
\end{itemize}

Cloud computing podría definirse como un modelo para permitir acceso conveniente, a demada y a través de la red a un conjunto compartido y configurable de recursos computacionales que pueden ser rápidamente proveídos y lanzados con un mínimo esfuerzo de administración o interacción con el proveedor.

Distinguimos dos categorías principales de soluciones bajo este paradigma: Infrastructure-as-a-Service (IaaS) y Platform-as-a-Service (PaaS), caracterizadas a continuación:

\begin{description}
	\item[IaaS:] El proveedor del servicio ofrece un lote de máquinas virtuales con privilegios de root y una consola para administrar las mismas, desde donde se puede dar de alta nuevas máquinas, dar de baja máquinas existentes y ajustar los recursos de cada una como se desee. El proveedor abstrae al usuario de las tareas de mantenimiento, backups y demás del hardware sobre el que corren las máquinas virtuales, y el usuario es responsable de la administración del software que corre sobre las mismas.
	
	En el contexto de una solución MongoDB, es responsabilidad del usuario montar y administrar su propio cluster MongoDB.

	\item[PaaS:] El proveedor ofrece un servicio determinado listo para ser utilizado por el cliente y una consola desde la que se pueden ajustar distintos parámetros del mismo, administrar los datos almacenados en éste, etc. El proveedor se encarga de la administración del software y hardware sobre el que éste corre, abstrayendo totalmente al usuario de estos detalles.

	En el contexto que nos interesa, se dispondría de un cluster MongoDB completamente administrado por el proveedor, al cual el usuario accedería conectando su aplicación a un determinado servidor en internet y/o por medio de la consola de administración.
\end{description}


\subsection{Alternativas de \emph{IaaS} y \emph{PaaS}}

A continuación describimos brevemente algunos de los servicios más populares actualmente en el mercado.


\subsubsection{Amazon EC2}

Amazon Elastic Cloud Computing, o EC2, es uno de los proveedores \emph{IaaS} más avanzados del mercado. Sus principales características:

\begin{itemize}
	\item Permite customizar profundamente los parámetros de las máquinas virtuales, ya sea memoria RAM, tipo de CPU, cantidad de núcleos, tipo y espacio de almacenamiento, sistema operativo, software preinstalado, hardware especial como GPUs y placas de red de tasas de entrada/salida específicas, etc.

	\item Permite definir \emph{templates}, es decir, imágenes de distintas máquinas virtuales, a partir de las cuales crear nuevas instancias cuando se necesita aumentar la cantidad de nodos en el sistema del cliente.

	\item Permite escalar instantáneamente el sistema del cliente instanciando decenas, centenas o miles de máquinas virtuales desde una consola administrativa accedida desde la web, o automáticamente a través de una API.

	\item Permite elegir la ubicación geográfica de las máquinas virtuales entre datacenters ubicados en distintas ciudades y países para agregar redundancia al sistema y reducir el impacto de desastres naturales.

	\item Funciones adicionales como balanceo de carga automático, la posibilidad de ofertar en remates de horas de procesamiento no utilizadas en algunas máquinas virtuales ociosas, solicitud de números de IP de forma dinámica, monitoreo de la salud de las máquinas virtuales, etc.

	\item Integración con otros servicios de Amazon Web Services como almacenamiento masivo de datos, redes de contenido, caches, bases de datos, etc.
\end{itemize}

La ventaja más grande de Amazon EC2 es el nivel de control que uno tiene sobre casi cualquier detalle imaginable de las máquinas virtuales y la infraestructura en la que éstas yacen. Esto resulta ideal para proyectos grandes donde pequeños ajustes representan grandes ahorros en costos y mejoras en performance.

Por otro lado, el nivel de complejidad y la cantidad de decisiones que hacen falta tomar a la hora de desplegar una aplicación en Amazon EC2 hace que resulte poco conveniente para proyectos pequeños, como aplicaciones web que se utilizan una única vez o emprendimientos que recién comienzan.


\subsubsection{Rackspace}

Es otro proveedor de \emph{Iaas} muy popular que ofrece una cantidad enorme de planes armados para poder elegir uno que se acomode a las necesidades sin problema alguno. Permite elegir muchos detalles de las VM's a solicitar, incluyendo la posibilidad de tener el storage sobre SSD, y varias opciones de SO. También se puede elegir tener hardware dedicado en lugar de virtual, lo cual puede significar un aumento de performance por un costo adicional. También ofrece tener una nube híbrida con parte del hardware dedicado y el resto en una nube pública para compartir el acceso.

Rackspace utiliza interfaces open source para administrar los servicios proveídos, con lo cual no es necesario tener que aprender a usar contenido propietario, ni depender del mismo.

Un punto fuerte de Rackspace es su soporte 24\times7\times365.


\subsubsection{Linode}

Linode ofrece diferentes planes, los cuales van desde \$20 a \$800 mensuales. Los planes varían en cantidad de RAM, almacenamiento, datos para transferencia y prioridad de ejecución. El poder de procesamiento de los planes es igual en todos, pero tienen menor prioridad de ejecución que otros según el plan contratado.

Además de buenos planes, Linode ofrece tanto su implementación de balanceo de nodos inteligentes como la posibilidad de configurar el balanceo manualmente, una API para realizar tareas diversas de administración, soporte para múltiples imágenes con configuraciones individuales, IP's múltiples y la posibilidad de elegir entre distintos data centers en distintos lugares para reducir la latencia.


\subsubsection{MongoHQ}

MongoHQ es una solución \emph{PaaS} que ofrece un cluster MongoDB completamente administrado por el proveedor. Sus principales características son:

\begin{itemize}
	\item Permite customizar la capacidad de almacenamiento del cluster y el tipo de almacenamiento (HDD vs SSD.)

	\item Ofrece una consola web para administrar el cluster, hacer consultas, realizar backups, etc.

	\item Ofrece herramientas de monitoreo y control de sanidad adicionales que normalmente no están disponibles en una instalación MongoDB por defecto.

	\item Permite escalar automáticamente el cluster sin downtime, agregando almacenamiento o nodos adicionales de acuerdo a lo necesario.

	\item Ofrece planes tanto para emprendimientos pequeños como para proyectos grandes.
\end{itemize}

Es un servicio ideal para equipos que necesitan disponer rápidamente de un cluster MongoDB sin perder tiempo en instalar y administrar un cluster ellos mismos, para proyectos que desean evaluar MongoDB o prototipar una solución sin grandes inversiones de tiempo y esfuerzo y para emprendimientos que necesitan salir rápido al mercado.

Su principal desventaja es que es menos flexible que un servicio que ofrece más granularidad y que se encarece rápidamente a medida que crece el volúmen de datos procesados comparado con mantener un cluster propio sobre una solución \emph{IaaS} como Amazon EC2. Por estos motivos MongoHQ no es una opción recomendable para proyectos grandes.


\subsubsection{MongoLab}

Este proveedor ofrece dos tipos de servicio para nuestra solución: sobre hardware compartido o dedicado. Aquí podemos elegir según nuestras preferencias, teniendo en cuenta que sobre las VM's dedicadas tenemos más libertades y más garantías. El hosting en todos los casos se realiza con alguno de los proveedores de infraestructura maś destacados del mercado (Amazon, Rackspace, Microsoft, y otros).

MongoLab tiene muchos planes prearmados para elegir, y nos permite elegir sobre cuál de los distintos proveedores con los cuales trabajan queremos hostear nuestra solución. Esto nos permitiría tener nuestros datos en MongoDB en los equipos del mismo proveedor que otras potenciales aplicaciones.

Por otra parte, para poder disponer del espacio de almacenamiento que necesitamos, tendríamos que recurrir a la opción de hardware dedicado, lo cual involucra precios mucho más elevados, pero también muchas menos preocupaciones.


\subsection{Proveedor recomendado}

Para nuestra aplicación, la mejor elección sería un servicio PaaS. Al utilizar como plataforma MongoDB (solo necesitamos utilizar querys sobre MongoDB), un servicio PaaS brinda la posibilidad de cargar nuestros scripts sin pérdida de tiempo alguna, evitando el mantenimiento y configuración del software necesario para correr una máquina virtual como pasaría con otros servicios.

Además de los beneficios de tener una plataforma lista, el servicio PaaS ofrece monitoreo de las bases de datos y escalabilidad costeable.


\subsection{De la nube al hardware propio}

El paso de una solución en la nube a una basada en hardware propio puede traer consigo muchos inconvenientes. Tenemos que tener en cuenta no sólo los inconvenientes generales que involucra este cambio, si no también los específicos relacionados con la solución sobre MongoDB.

En principio hay que tener en cuenta que al momento de hacer el cambio definitivo, en caso de no tomar los recaudos necesarios, puede existir un downtime del servicio, el cual es súmamente indeseable. Esto se puede evitar por ejemplo con la implementación de un balanceador de carga que distribuya el tráfico gradualmente desde la infraestructura en la nube hacia nuestros servidores locales; una vez que el tráfico sea completamente soportado por nuestros servidores, puede darse de baja el sistema del proveedor.

Algo muy importante es el costo que implica este cambio: la adquisición de todo el equipamiento necesario, que no sólo consiste en los servidores en sí, si no también en el lugar físico del data center, sistemas de refrigeración, personal de mantenimiento y seguridad, y una larga lista de etcéteras. Este costo se puede amortizar en varios años, pero de todas formas es una inversión muy grande que se realiza en un lapso corto de tiempo, y se pierde la constancia que nos ofrece un servicio en la nube con precios que escalan de forma gradual. Y no sólo es un problema de costos del equipamiento adquirido, también hay que encargarse de que el mantenimiento del mismo se haga de manera correcta, un problema del que antes no teníamos que preocuparnos.

Para analizar el caso particular de nuestra implementación, tenemos que tener en cuenta que para aprovechar MongoDB nos conviene tener un crecimiento horizontal en función de la cantidad de datos a procesar. En el caso de la solución en la nube, era algo muy simple de implementar, pues las VM's se pueden crear a demanda en cuestión de segundos y de forma transparente. Pero cuando se tiene hardware propio hay que tener en cuenta que el crecimiento de la infraestructura no es tan simple de lograr; se pueden tener los recursos computacionales virtualizados, pero será necesario adquirir más hardware eventualmente.

También hay que considerar que en algunos momentos, el procesamiento puede alcanzar picos durante los cuales sea necesario aumentar la capacidad de procesamiento del sistema. Esto es algo que también sucede de forma transparente en caso de ser necesario cuando estamos trabajando en la nube, pero no así cuando tenemos nuestros propios data centers. Nuevamente, podemos tener los recursos virtualizados, pero el hardware tiene un límite y es algo que ya no podemos obviar.


\end{document}